
%%% Local Variables:
%%% mode: latex
%%% TeX-master: t
%%% End:

\chapter{总结:未来数据中心}
\label{cha:pardimpl}

互联网应用如电子邮件、搜索、网络购物、社交网络、在线视频、网络地图等, 已经成为人们
生活的一部分。这些应用往往要为上亿用户服务,意味着互联网应用已变成如电力一样的社会
公共服务,而支撑拥有海量用户互联网应用的数据中心也成为如同发电厂一样的社会核心基础
设施。

典型的数据中心一般有5~10万台中低端服务器组成,这些服务器通过内部网络互连,一起协同
运行互联网应用为海量用户服务。因为这类数据中心规模很大,往往部署在大型仓库级别的机
房,从应用角度来看就如同一台计算机,因此也被称为“仓库级计算机(Warehouse-Scale Computer)” [1]。国
内外著名的互联网公司往往拥有多个数据中心,服务器数量达到数十万甚至上百万台。例如,
谷歌(Google)的数据中心服务器数量已经超过百万台为全球用户提供搜索、邮件、地图等服务[2];
亚马逊(Amazon)仅EC2就部署了约50万台服务器提供云计算服务[3];据可靠消息,国内腾讯
公司也拥有约30万服务器为用户提供各种互联网服务。

尽管目前互联网企业的数据中心已经颇具规模,但一个趋势是未来数据中心还将持续发展。一
方面互联网用户数量仍在不断增长,目前全球已有24亿网络用户,但很多机构预测未来全球还
将新增30亿网民融入到互联网[4],这会对数据中心的数量和规模都提出更多需求。另一方面快
速发展的移动终端已超越个人计算机(PC),成为终端计算设备的主流。由于移动设备性能相对
较低、存储容量较小,将计算与存储转移到数据中心的需求也变得越来越强烈。
因此数据中心作为基础设施也会日益重要。

