
%%% Local Variables:
%%% mode: latex
%%% TeX-master: t
%%% End:

\chapter{标签化地址空间}
\label{chap:labeladdrspace}

虚拟地址空间对应多进程场景,随着单节点计算能力的增强以及云计算场景发展,
多租户场景成为一种趋势。

虚拟机抽象的出现是应对这下趋势的一个变化,将一台物理机隔离为多个无关的虚拟机,
现在虚拟机基于的硬件技术包括:内存划分技术EPT、I/O虚拟化I/O-MMU、MSI中断等。
这些技术在Hypervisor的支持下协同工作,向上层展现出虚拟的隔离计算机。

但这里的隔离并非真正的隔离,多个虚拟机仍然通Hypervisor实现对共享硬件资源的使用,
如hypervisor需要对页表寄存器进行控制,实现地址空间的切换;
通过对I/O设备的代理访问,实现虚拟I/O设备;

标签化地址空间的目的是将多个虚拟机真正的物理隔离开。

%%
%%Why we need PARD? => single/batch-program change to multi-thread program to multi-tent/cloud
%%但体系结构几乎没有变化,不能区分不同应用,造成\ref{chap02}中的一些问题。
%%随着核数增多,单个应用无法用满所有的核,出现了虚拟化技术,其本质是在虚拟地址空间之外增加一层地址空间
%%本章介绍PARD体系结构与其基本特性,对比PARD与SDN结构,PARD系统使用实例,如何在现有系统中构造出PARD。
%%% Need of a PARD's labeled address space
%%\section{标签化地址空间}
%%%软件层标签化(cgroup/VM),逻辑域,
%%%标签化用途
%%%如何实现
%%多个虚拟地址空间对应一个物理地址空间。
%%EPT增加了一级地址映射,实现内存划分;I/O MMU技术让硬件设备能够识别EPT增加的一级地址空间。
%%PCI-E的SR-IOV技术让设备
%%


\section{标签与隔离粒度}

如何划分地址空间:虚拟机、容器、进程、etc...

\section{标签的传播}

不同总线上如何实现地址空间

\section{通过标签实现内存地址空间隔离}

地址映射位置,与EPT技术对比

Cache一致性场景下如何实现标签化地址空间

处理器核共享场景(0.1个处理器核)

\section{通过标签实现I/O地址空间隔离}

SR-IOV or pseudoMR-IOV

