
%%% Local Variables: 
%%% mode: latex
%%% TeX-master: t
%%% End: 

\chapter{相关研究}
\label{chap:related}

应用混合的目标分为两方面:一是提高资源利用率,二是保障关键应用的服务质量。现有的运行时
管理方案[15]–[17]大都通过硬件性能计数器对关键应用的性能进行监控,并在性能发生下降时对非
关键应用进行各种处理,以减小由于资源竞争引起的性能问题。

造成资源利用率的核心问题在于计算机资源处于“无管理共享”状态,因此多个应用共享资源时会发
生竞争与干扰,最终导致关键应用性能不可预测。目前尚无很好的技术方案解决计算机资源“无管理
共享”的问题,以Google为代表的工业界采用将在线服务器与离线服务器分离的方法,通过降低在线
服务器的负载来保障在线应用的服务质量。

学术界在共享资源管理方面从2个维度、4个方面开展研究:软件调度、软件划分、硬件调度、硬件
划分。软件调度是通过操作系统/Hypervisor层次进行进程、线程或虚拟机级别的调度,一般调度粒
度较大,需要上下文切换,时间上也需要几毫秒到几十毫秒,不能满足在线应用的快时响应的需求;
软件划分技术能对Cache容量进行划分,但无法管理访存带宽这类资源,且软件划分技术配置调整开
销较大;硬件调度技术能支持访存请求级别的细粒度调度,但灵活性较多,不能根据不同应用按需管
理;硬件划分技术对Cache容量比较有效,但也无法对带宽等进行管理,而且同样面临灵活性差的问
题。面对诸多问题,斯坦福大学Christos Kozyrakis教授提出应该重新考虑整个计算机架构,从应用
特征、硬件隔离、操作系统、机群调度、高效资源管理硬件等多层次协同设计[18]。

\section{调度方法}
\label{sec:other}


\section{隔离方法}
\label{sec:multifig}


\section{本章小结}

从现有技术来看,单节点内服务质量保障技术的不足,导致节点内应用相互干扰严重,某种程
度上成为目前数据中心整体服务质量保障的短板,是成为长尾延迟现象的主要因素之一。同时
这也是一个非常具有挑战的问题,这需要跨层次协同设计。美国计算共同委员会(Computing 
Community Consortium)于2012年5月发布的计算机体系结构共同体白皮书《21世纪计算机体系
结构》中也将单节点内保障服务质量作为未来研究方向之一,其中认为[9]:“管理应用之间的相
互作用也带来了挑战。例如,这些应用如何表达服务质量(QoS)目标并且让底层的硬件、操作
系统以及虚拟层共同工作来保障它们。”

