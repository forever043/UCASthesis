
%%% Local Variables:
%%% mode: latex
%%% TeX-master: t
%%% End:
\secretcontent{绝密}

\ctitle{面向数据中心服务质量的可编程体系结构}
\makeatletter
\makeatother
\cdegree{工学博士}
\cdepartment[计算所]{中国科学院计算技术研究所}
\cmajor{计算机科学与技术}  %写一级学科名称
\cauthor{马久跃}
\csupervisor{孙凝晖\hspace{1em}研究员}
\csupervisorplace{中国科学院计算技术研究所}

% 日期自动生成,如果你要自己写就改这个cdate
% \cdate{\CJKdigits{\the\year}年\CJKnumber{\the\month}月}

\etitle{Programmable Architecture for Quality-of-Service in Datacenters }
\edegree{Doctor of Philosophy}
\eauthor{Ma Jiuyue}
\edepartment{Institute of Computing Technology, Chinese Academy of Sciences}
\emajor{Computer Science and Technology}   %写一级学科名称
\esupervisor{Sun Ninghui}

% 这个日期也会自动生成,你要改么?
% \edate{December, 2005}

% 中英文摘要和关键字
\begin{cabstract}
  当前数据中心正面临着提高资源利用率与保障应用服务质量的挑战。
  通过负载融合的方式将来自不同用户的应用共享服务器资源,可以有效的提高资源利用率,
  但这种无管理的软硬件资源共享所产生的资源竞争,会给应用带来不可预测的性能波动。
  为了保障延迟敏感型在线应用的服务质量,数据中心系统会倾向于避免共享,
  使用独占或过量资源预留的方式降低由于共享对在线应用的影响,
  造成了数据中心6\%$\sim$12\%极低的资源利用率。

  该问题产生的根本原因是当前计算机体系结构不能实现应用区分,
  无法实现应用之间的性能隔离,使得共享资源的应用之间产生干扰。
  因此,在应用数量众多、需求多样且不断变化的数据中心场景下,
  计算机体系结构需要重新设计,为应用提供区分化服务、良好的性能隔离,
  并具备灵活的资源管理编程接口,实现资源使用的强控制与按需分配,
  才能解决数据中心资源利用率与服务质量冲突的问题。

  本文围绕数据中心资源利用率与应用服务质量冲突的问题,
  在服务质量保障的体系结构支持方向上开展研究,主要的工作和贡献包括:

  1)提出``标签化地址空间(Labeled Address Space)''概念,用于实现应用区分。
     正如多进程技术的出现引入了虚拟地址空间抽象,随着虚拟化、云计算与多租户使用模式的出现,
     现有的虚拟地址空间抽象无法满足多租户之间的隔离需求,
     一些硬件隔离技术如EPT、I/O MMU、SR-IOV等试图在现有的体系结构下支持隔离需求,
     其本质则是在虚拟地址空间外增加一层额外的地址空间,
     这些技术只是在功能层面上实现了隔离,
     而与性能相关的部件如共享末级缓存、内存控制器等在现有体系结构下并没有实现隔离。
     本文提出的标签化地址空间抽象,使用统一标签区分不同应用,
     为计算机系统内所有请求标识应用标签,硬件不再需要通过猜测的方式区分应用,
     使得共享的硬件资源能够区分来自不同应用的请求并进行区分处理。

  2)提出可编程硬件资源共享管理方法,实现节点内硬件资源的统一管理。
     通过将硬件资源抽象为``控制平面''与``数据平面'',
     使用软件编程的方式对硬件资源进行统一管理,
     包括基于``控制平面''的策略可编程和``数据平面''的机制可编程。

  3)提出本地资源带外管理的方式,将业务与资源管理分离,
     设计可编程体系结构硬件资源管理的软件接口,实现本地资源的灵活管理。
     利用Linux的sysfs机制,在集中式的资源管理模块中将``控制平面''和
     ``数据平面''抽象为文件,为用户提供基于文件的硬件资源编程接口。
     同时提出了``\emph{trigger$\Rightarrow$action}''机制实现硬件资源的实时监控与反馈调节,
     以及不同资源之间的协同管理。

  4)构建资源管理可编程体系结构实验平台,包括基于gem5的模拟器与FPGA原型系统,
     其中模拟器已经在LGPL协议下开源。
     FPGA原型系统验证结果表明,通过标签化地址空间与可编程硬件资源共享管理方法,
     PARD体系结构能够为计算机带来应用区分化服务、性能隔离与资源管理可编程特性,
     同时也不会在系统中引入过大的性能开销与资源开销。
     基于以上这些硬件机制与软件栈及适当的资源管理策略,
     PARD体系结构能够进行高效的资源管理,实现数据中心资源利用率与应用服务质量的平衡。
\end{cabstract}

\ckeywords{数据中心, 服务质量, 资源利用率, 虚拟化, 资源管理, 可编程}

\begin{eabstract}
  Contemporary data centers confront with challenges in
  managing the trade-offs between resource utilization and
  applications’ quality of services (QoS). 
  Co-locate multiple workloads into a single server can
  achieve high utilization.
  But the unmanaged contention for shared hardware resources,
  as well as shared software resources,
  may cause unpredictable performance variability.
  To guarantee QoS of latency-critical online services,
  data center operators or developers tend to avoid sharing by
  either dedicating resources or exaggerating reservations
  for online services in shared environments,
  which result in lower utilization, only 6\% to 12\%.

  Unable to distinguish different applications,
  which makes it is impossible to achieve the performance isolation between applications for
  current computer architecutre 
  Current computer architecture can not distinguish different applications,
  which make it unable to achieve the performance isolation between applications.
  The key reason root cause of the problem is that the current computer architecture 
  cannot distinguish applications, furthermore cannot do performance isolation 
  which results in contentions on shared resources among applications. 
  Due to the large amount of applications of diverse demands 
  in data center, the computer architecture needs to be redesigned 
  to support differentiated services, better performance isolation, 
  flexible programming interfaces, strong control over resources and allocation on demand.
  
  Focuses on the trade-off between resource utilization and applications'
  quality of services, this dissertation studies the architecture for quality of service
  in data centers. The main works and contributions include:

  (1) A ``Labeled Address Space'' abstraction is proposed
      for application distinguish purpose.
      As the emergence of virtualization and cloud computing technologies,
      the requirements of multi-tenant isolation can not be fulfiled by the virutal
      address space abstraction introduced from multi-process.
      Lots of hardware mechanisms, such as EPT, I/O MMU and SR-IOV,
      are proposed to enhance the existing architecture for better performance isolation.
      Essentially, these mechanisms try to add an extra layer to existing virtual address
      space, which successfully achieve the functional isolation.
      But these performance related resources,
      such as shared last level cache, memory controller,
      are not isolated at all.
      By tagging all the requests in computer with an uniformed label introduced by
      labeled address space, the shared hardware-resources can support
      differentiated service according to the tagged label.

  (2) A unified resource management method which makes ``control plane'' and
      ``data plane'' abstractions for shared hardware resources.
      It manages all shared hardware resources in a programmable manner through
      strategy programmable offered by control plane and
      mechanism programmable realized by data plane.

  (3) A out-of-band resource management architecture,
      which seperate management from business.
      It expose the programmable resource management abilities of hardware through
      sysfs mechanism provided by linux.

  (4) A prototype of the proposed architecture PARD, including a gem5-based simulator
      and a FPGA prototype.
      

\end{eabstract}

\ekeywords{datacenter, QoS, utilization, virtualization, resource management, programmable}

