
%%% Local Variables:
%%% mode: latex
%%% TeX-master: t
%%% End:

\chapter{引言}
\label{chap:intro}

互联网应用如电子邮件、搜索、网络购物、社交网络、在线视频、网络地图等, 已经成为人们
生活的一部分。这些应用往往要为上亿用户服务,意味着互联网应用已变成如电力一样的社会
公共服务,而支撑拥有海量用户互联网应用的数据中心也成为如同发电厂一样的社会核心基础
设施。

服务质量(QoS)与资源利用率是数据中心运营时需要考虑的两个重要指标,前者严重影响用户
体验,而后者直接与数据中心的运营成本相关。然而现有的计算机体系结构并没有为服务质量
保障提供足够的支持,造成这两个指标在现实状况下存在冲突。为了保障用户体验,在实际系
统部署时,会更多的考虑服务质量这一指标,造成数据中心的资源利用率严重低下,普遍只有
10\%-30\%左右。基于这一现状,本文主要讨论如何设计一种高效的数据中心体系结构,使得
数据中心在保障应用服务质量基础上,达到较高的资源利用率。

本章内容安排如下:首先介绍新计算模式对数据中心的挑战,然后讨论现有数据中心技术的局
限性,即服务质量与资源利率冲突的原因,进而提出一种面向数据中心应用服务质量保障的
体系结构,最后将阐述本文的研究动机,介绍本文的主要贡献和组织结构。


\section{新计算模式对数据中心的挑战}

典型的数据中心一般有5~10万台服务器组成,建设与运行维护成本往往高达几十亿人民
币。然而出于保障应用服务质量的原因,现有数据中心只能维持较低的资源利用率,导致大量
资源浪费。因此,本项目总体研究目标为如何设计高效通用数据中心体系结构:“通用”表
示数据中心可同时运行各种不同类型应用;“高效”表示数据中心能在保障延迟敏感应用的服
务质量基础上,达到较高的资源利用率(CPU利用率>60\%)。

典型的数据中心一般有5~10万台中低端服务器组成,这些服务器通过内部网络互连,一起协同
运行互联网应用为海量用户服务。因为这类数据中心规模很大,往往部署在大型仓库级别的机
房,从应用角度来看就如同一台计算机,因此也被称为
“仓库级计算机(Warehouse-Scale Computer)“ \cite{WSC}。
国内外著名的互联网公司往往拥有多个数据中心,服务器数量达到数十万甚
至上百万台。例如,谷歌(Google)的数据中心服务器数量已经超过百万台为全球用户提供
搜索、邮件、地图等服务[2];亚马逊(Amazon)仅EC2就部署了约50万台服务器提供云计算服
务[3];据可靠消息,国内腾讯公司也拥有约30万服务器为用户提供各种互联网服务。

尽管目前互联网企业的数据中心已经颇具规模,但一个趋势是未来数据中心还将持续发展。一
方面互联网用户数量仍在不断增长,目前全球已有24亿网络用户,但很多机构预测未来全球还
将新增30亿网民融入到互联网[4],这会对数据中心的数量和规模都提出更多需求。另一方面快
速发展的移动终端已超越个人计算机(PC),成为终端计算设备的主流。由于移动设备性能相对
较低、存储容量较小,将计算与存储转移到数据中心的需求也变得越来越强烈。因此数据中心作
为基础设施也会日益重要。


\section{现有数据中心技术的局限性}

通过上述分析可知,移动计算与实时计算均对快速响应用户请求提出了强烈的需求。而当前数
据中心为了保障用户请求的服务质量,不得不通过采用牺牲资源利用率、保留过量资源的方式。
Google的数据中心技术一直处于领先地位,我们以Google为例分析数据中心资源利用率现状。
图5显示了2006年Google数据中心平均CPU利用率为30\%左右。但到2013年,虽然Google将数据
中心分为了两类,并且批处理数据中心已经能达到75\%的CPU利用率,但在线应用数据中心仍停
留在30\%。我们对国内企业调研发现,几大主流互联网企业在线应用数据中心CPU利用率一般都
低于20\%,有的甚至低于10\%,仍然存在很大的提升空间。

\section{资源管理可编程体系结构}
aaa

\section{本文的研究动机}
aaa

\section{本文的主要贡献}
aaa

\section{论文的组织结构}
aaa

