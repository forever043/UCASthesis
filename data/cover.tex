
%%% Local Variables:
%%% mode: latex
%%% TeX-master: t
%%% End:
\secretcontent{绝密}

\ctitle{面向数据中心服务质量的可编程体系结构}
\makeatletter
\makeatother
\cdegree{工学博士}
\cdepartment[计算所]{中国科学院计算技术研究所}
\cmajor{计算机科学与技术}  %写一级学科名称
\cauthor{马久跃}
\csupervisor{孙凝晖\hspace{1em}研究员}
\csupervisorplace{中国科学院计算技术研究所}

% 日期自动生成,如果你要自己写就改这个cdate
% \cdate{\CJKdigits{\the\year}年\CJKnumber{\the\month}月}

\etitle{Programmable Architecture for Quality-of-Service in Datacenters }
\edegree{Doctor of Philosophy}
\eauthor{Ma Jiuyue}
\edepartment{Institute of Computing Technology, Chinese Academy of Sciences}
\emajor{Computer Science and Technology}   %写一级学科名称
\esupervisor{Sun Ninghui}

% 这个日期也会自动生成,你要改么?
% \edate{December, 2005}

% 中英文摘要和关键字
\begin{cabstract}
  当前数据中心正面临着提高资源利用率与保障应用服务质量的挑战。
  负载融合是提高服务器利用率的主要方式,
  将不同用户的应用部署在同一台服务器,通过资源共享的方式能够提高资源利用率。
  但当前计算机中无管理的软硬件资源共享所产生的资源竞争,
  会给应用带来不可预测的性能波动。
  为了保障延迟敏感型在线应用的服务质量,数据中心系统会倾向于避免共享,
  使用独占或过量资源预留的方式降低由于共享对在线应用的影响,
  造成了数据中心6\%$\sim$12\%极低的资源利用率。

  硬件不能区分来自不同应用的请求,无法实现应用之间的性能隔离,
  使得共享资源的应用之间产生干扰,是数据中心问题的根源。
  因此,在应用数量众多、需求多样且不断变化的数据中心场景下,
  计算机体系结构需要重新设计,为应用提供区分化服务、良好的性能隔离,
  并具备灵活的资源管理编程接口,实现资源使用的强控制与按需分配,
  才能解决数据中心资源利用率与服务质量冲突的问题。

  本文围绕数据中心资源利用率与应用服务质量冲突的问题,
  在服务质量保障的体系结构支持方向上开展研究,主要的工作和贡献包括:

  1)提出``标签化地址空间(Labeled Address Space)''概念,用于实现应用区分。
     基于进程的虚拟地址空间抽象无法满足数据中心多租户应用之间的隔离需求,
     一些硬件隔离技术如EPT、I/O MMU、SR-IOV试图扩展现有的体系结构,
     但它们本质上只在功能层面上实现了隔离,
     与性能相关的部件如共享末级缓存、内存控制器等由于缺少应用信息,
     依然处理于无序共享的状况,无法实现性能隔离。
     本文所提出的标签化地址空间抽象,使用统一标签区分不同应用,
     通过在计算机系统内所有请求标识应用标签,让硬件能够实现应用识别与区分化处理,
     实现性能隔离。

  2)提出通用硬件资源管理方法与架构。
     通过统一的资源管理接口实现不同类型硬件上机制与策略的统一管理,
     使用软件编程的方式根据应用需求对硬件的机制与策略进行调整,
     包括基于``控制平面''的策略可编程和``数据平面'' 的机制可编程,
     以支持数据中心复杂多变的应用场景。

  3)提出本地资源带外管理的方式,实现资源管理与业务逻辑的分离。
     在集中式的资源管理模块中,利用Linux的sysfs机制将``控制平面''和
     ``数据平面''抽象为文件,为管理员提供基于文件的硬件资源编程接口。
     实现本地资源的灵活管理。
     同时提出了``\emph{trigger$\Rightarrow$action}''机制实现硬件资源的实时监控与反馈调节,
     以及不同资源之间的协同管理。

  4)构建资源管理可编程体系结构实验平台,包括基于gem5的模拟器与FPGA原型系统,
     其中模拟器已经在LGPL协议下开源。
     FPGA原型系统验证结果表明,通过标签化地址空间与可编程硬件资源共享管理方法,
     PARD体系结构能够为计算机带来应用区分化服务、性能隔离与资源管理可编程特性,
     同时也不会在系统中引入过大的性能开销与资源开销。
     基于以上这些硬件机制与软件栈及适当的资源管理策略,
     PARD体系结构能够进行高效的资源管理,实现数据中心资源利用率与应用服务质量的平衡。
\end{cabstract}

\ckeywords{数据中心, 服务质量, 资源利用率, 虚拟化, 资源管理, 可编程}

\begin{eabstract}
  Contemporary data centers confront with challenges in
  managing the trade-offs between resource utilization and
  applications’ quality of services (QoS). 
  Co-locate multiple workloads into a single server can
  achieve high utilization.
  But the unmanaged contention for shared hardware resources,
  as well as shared software resources,
  may cause unpredictable performance variability.
  To guarantee QoS of latency-critical online services,
  data center operators or developers tend to avoid sharing by
  either dedicating resources or exaggerating reservations
  for online services in shared environments,
  which result in lower utilization, only 6\% to 12\%.

  Unable to distinguish different applications and achieve performance isolation,
  resource contention in existing computer architecture is the root cause of
  the data center problem.
  Due to the large amount of applications of diverse demands in data center,
  the computer architecture needs redesign to resolve the problem
  by supporting differentiated services, better performance isolation,
  flexible programming interfaces, strong control, and resource on demand.
  
  Focuses on the trade-off between resource utilization and applications'
  quality of services, this dissertation studies the architecture for quality of service
  in data centers. The main works and contributions include:

  (1) A ``Labeled Address Space'' abstraction is proposed
      for applications distinguish purpose.
      %As the emergence of virtualization and cloud computing technologies,
      The virtual address space abstraction introduced by multi-process
      can not support isolation of multi-tenant.
      Recently, some hardware mechanisms, such as EPT, I/O MMU and SR-IOV,
      are proposed to enhance the existing architecture for better isolation.
      Essentially, these mechanisms only achieve functional isolation,
      which means the performance related resources,
      such as shared last level cache, memory controller, are still unmanaged.
      Contention caused by unmanaged sharing makes performance isolation unpractical.
      By tagging all the requests in computer with a uniformed label introduced by
      labeled address space, the shared hardware-resources can support
      differentiated service according to the tagged label,
      which brings the benefit of performance isolation.

  (2) An unified resource management method and architecture is proposed,
      which makes ``control plane'' and ``data plane'' abstractions for
      shared hardware resources.
      It manages all shared hardware resources in a programmable manner through
      control plane based strategy programmable and
      mechanism programmable realized by data plane.
      The programmable characteristic makes it more adapted to the complex
      data center scenario.

  (3) An out-of-band resource management architecture is proposed,
      which separate resource management from business.
      In the per-node centralized resource management module, 
      the ``control planes'' and ``data planes'' are abstract as files
      through sysfs mechanisms provided by the linux-based firmware.
      Data center operator can achieve flexible resource management through 
      this file-based interface.
      A ``\emph{trigger$\Rightarrow$action}'' mechanisms is also provided to implement
      real-time resource monitoring and adjustment, and joint management of different
      hardware resources.

  (4) A prototype of the proposed architecture PARD is proposed,
      including an open-sourced gem5-based simulator and a FPGA prototype.
      The results from FPGA prototype shows that, the PARD architecture brings
      the features of differentiated service, performance isolation,
      and programmable resource management with only little resource overheads.
      With these features, PARD can realize efficient resource management and
      achieve balance of the resources utilization and
      applications' quality of service in data center.



\end{eabstract}

\ekeywords{datacenter, QoS, utilization, virtualization, resource management, programmable}

