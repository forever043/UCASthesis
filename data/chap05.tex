
%%% Local Variables:
%%% mode: latex
%%% TeX-master: t
%%% End:

\chapter{PARDg5-V:一个PARD的模拟器实现}
\label{cha:pardg5v}

上一章已经讨论过如何在X86平台上实现PARD,本章讨论如何在gem5全系统模拟器上将PARD机制
实现其中。验证功能可行性,

\section{基本配置}

\section{模拟器设计目标}

与网络/SDN对比的功能表格,列出实现目标

列出模拟器局限

\section{系统设计与关键技术}

\subsection{实现全硬件虚拟化}

标签机制 + core/LLC/mc/io
ptable + MultiOS

\subsection{实现资源预留}

stable统计 + LLC partition

\subsection{实现QoS}

ttable + scheduling + feedback

\subsection{管理}

PRM + Program/Trigger

\subsection{小结}

总结已实现的功能,并对未来可能扩展的功能进行展望

\section{评估}

\section{小结}


