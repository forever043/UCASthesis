
%%% Local Variables:
%%% mode: latex
%%% TeX-master: t
%%% End:

\chapter{硬件资源共享管理方法}
\label{chap:hwresman}

% 现有体系结构共享性能很差
由于缺少有效的硬件资源共享管理机制,现有服务器体系结构不能很好的在共享场景中工作。
不同应用无管理的访问共享硬件资源,产生资源竞争并造成严重的性能干扰,进而影响应用性能,
来自Intel的研究\cite{intel}表明在共享末级缓存上的资源竞争最多可造成30\%的性能干扰。
% 而现实需求又需要共享
随着计算机硬件技术不断发展,现在计算机所能提供的硬件资源不断增加,
单个应用很难充分利用全部硬件资源;
与此同时,云计算与移动互联网的发展与普及,应用的数量也在飞速增加。
通过虚拟化与容器等技术将不同应用运行在同一台服务器,提高服务器资源利用率,
应用共享硬件资源成为常态。
% 资源利用率与服务质量相冲突
%但应用服务质量与服务器资源率之间存在冲突,
在这种场景下为了保障延迟敏感型应用的性能不受影响,
这些延迟敏感型应用需要独占部分共享资源(如部署在单独的服务器\cite{}),
这造成数据中心服务器的资源利用率通常只有10\%-12\%。
如何为硬件资源提供灵活的共享管理方法,是解决服务器多应用共享的关键问题。


% 问题一:资源竞争点不断变化,需要统一的管理接口,control plane + ?PRM?
现有一些研究工作在单一硬件资源上提出共享管理的方法,
如在共享末级缓存上提出容量划分的方法\cite{},
在内存控制器上提供请求调度的方法\cite{},
Intel在其Xeon E5v3系列处理器中增加了末级缓存容量划分功能。
但是计算机内包含大量的共享硬件资源,不同应用混合后,会在不同的硬件上资源上产生竞争;
同时由于硬件资源之间存在关联,需要协调管理。
例如仅通过容量划分的方式解决在共享缓存中的资源竞争,
可能会造成在竞争点转移到内存控制器,这是由于
因此,为所有的硬件资源提供统一的共享管理接口,
并集中式的管理节点内的全部共享资源是必要的。

%
要实现硬件资源的共享管理,首先需要考虑如何监控资源用量,以及实时反馈。 % stab+ttab
现有体系结构实现中已实现的监控大都是针对片内资源,如处理器核IPC、分支预测正确率、XXX等,
或缓存缺失率等信息,但这些信息都需要操作系统或库的支持,手动获取信息,无法实现实时的监控,
更无法实现实时反馈;而且这些信息通常是以处理器核做为区分,而无法直接映射到应用,
即无法直接获取应用相关的信息,需要软件进行预处理或后处理才能得到,通常这个时间是微秒甚至是秒级的;
当前数据中心需要高速的响应,因此如何实现毫秒级的资源监控与反馈是实现细粒度资源管理的基础。
本文所提出的控制平面中直接包含了用于收集硬件资源使用情况的硬件逻辑,
能够实现实时的性能监控,同时提供了性能条件触发逻辑,对资源使用变化进行实时的反馈。

对于硬件,如何实现干扰隔离。 % cp_proc

本文提出一种通用的硬件控制平面方法,为共享硬件资源提供共享资源管理的功能,具体包括:
基于表的资源管理抽象,实时性能监控与反馈,资源调整机制,可编程数据平面处理器。
与之前工作的主要区别是,(1)统一的资源抽象,而不是为单独某一个硬件资源;
(2)基于硬件的实时资源监控(3)

XXXX

xxx

XXXX

本章主要
%In this chapter I focus on the failure recovery, metadata I/O efficiency, and adaptability
%implications of the combined approach to metadata storage, journaling, and workload
%distribution and relate my experiences constructing a working implementation. I analyze a
%variety of static file system snapshots and workload traces to motivate MDS design and performance
%analysis, present a simulation based-analysis of metadata partitioning approaches to
%demonstrate the architectural advantages of a dynamic subtree-based approach, and evaluate
%my implementation under a range of micro-benchmarks, workload traces, and failure scenarios.


%这种多应用共享的使用方式在现有的服务器架构下会造成应用之间严重的干扰。
%为了防止这些硬件层次的干扰对延迟敏感型应用的性能造成影响,
%通常不会允许其它应用与延迟敏感型应用共享服务器资源,
%造成很低的服务器资源利用率。


\section{背景与相关工作}

现在的计算机中包括多种共享硬件资源,这些硬件又通过共享的数据通路相连,
需要一种共享资源管理机制来保障多应用共享时不会造成应用之间的干扰。
已有大量工作研究在不同的部件上支持多应用共享管理,本节将讨论其中一些工作,并分析共享管理中的难点。


\subsection{共享硬件资源管理的难点}
1,怎么算好的共享管理
2,需要什么样的接口
3,实时监控是个难点
4,全局调节是个总是(在下一章中解决)

\section{硬件资源抽象}

计算机内的硬件资源主要可以分为两类,一类是基于容量的资源,一类是基于流量的资源。
例如,共享缓存、处理器核、以及硬盘空间是基于容量的资源,

\section{控制平面设计}

%这里加一个图,说明一下控制平面与硬件的关系,就是大框套小框的关系
控制平面是在硬件之外增加的一层,对硬件的行为进行控制,
同时可以在必要时对发送到硬件的请求进行额外的处理,如地址变换、请求无效等。
由于控制平面能够感知到硬件处理的所有请求,因此可以实现实时性能监控与反馈,
由于控制平面能够对硬件行为进行控制,因此可以实现资源调整机制。

\subsection{实时资源监控与反馈}
资源监控是实现硬件资源可管理共享的第一步,要实现好的共享管理策略,必需对系统当前的资源分配情况、应用性能进行细粒度、实时的监控。
目前监控主要是软件和硬件两个方面,由于


\subsection{资源调整机制}

\subsection{控制平面微体系结构设计}

\subsection{控制平面示例}

\subsubsection{内存控制器控制平面}

\subsubsection{共享末级缓存控制平面}


\section{可编程数据平面设计}

在PARD中,计算机的硬件部件被抽象为数据平面与控制平面两部分,其中数据平面用于执行数据操作,而控制平面用于对数据平面的策略进行管理。以缓存控制器为例,数据平面中包含用缓存数据的DataArray与记录缓存内容的TagArray,以及替换策略的实现,其控制平面包含替换策略的参数、统计信息等;对于内存控制器而言,数据平面用于接收上层访存请求,并将其转换为内存地址,发送到内存阵列中,其控制平面用于控制地址映射与访存调度策略。从以上两个例子我们可以发现,PARD控制平面的可编程主要体现在其对数据平面所提供功能的控制,而数据平面所提供的功能在其设计完成后即已确定,如果需要增加额外的功能,则需要对数据平面进行重新的设计,并对控制平面暴露相应的接口。因此,控制平面的可编程能力受到数据平面所能提供功能的限制。

另一方面,PARD中的“triggeraction”机制在响应时间上存在瓶颈:每当触发条件发生后,需要经过控制平面网络将该事件传递到PRM,由运行在PRM中的软件代码来更新策略,并通过控制平面网络写回到控制平面中。由于受到控制平面网络延迟以及PRM软件代码延迟的影响,该机制并不能达到特别高的响应速度。同时并非所有的触发条件发生后都需要在PRM进行全局处理,完全可以预定义一些动作,在控制平面本地完成处理。
为解决以上两个问题,本文提出的可编程数据平面架构如图2所示。在PARD中所提出的控制平面/数据平面模型基础上,在数据平面中增加了多个可编程处理器,连接数据平面中其它的逻辑部分,并通过其中的固件代码对数据进行处理。控制平面依然使用三张控制表作为对外访问的接口,通过控制平面网络与集中式的平台资源管理模块通信。由于数据平面部件能够执行代码,因此可以使用软件代码来实现反馈调节,因此触发逻辑已经从控制平面中被移除。
 
要实现以上可编程数据平面架构,需要解决以下三个问题:(1)如何为不同硬件部件的数据平面增加处理器逻辑;(2)处理器如何设计,如何与控制平面进行通信;(3)处理器逻辑如何编程。本文后续章节将针对以上三个问题进行分别阐述。

\subsection{可编程数据平面}

在讨论数据平面处理器设计前,我们首先以内存控制器和缓存控制器为例,讨论可编程数据平面的设计。

\subsubsection*{内存控制器}
当前的处理器芯片通常会集成2个到4个独立的内存控制器,每个控制器使用独立的内存通道。每个内存通道连接到多个可并行访问的rank,而每个rank又是由多个共享地址与数据总线的二维存储阵列(bank)组成。内存控制器的主要工作就是接收上游的读写请求,并将其转换为下游的DRAM命令,完成数据传输。以内存控制器作为数据平面,可以在地址映射和访存调度两个位置增加可编程功能。

地址映射 地址映射分为两部分,首先是通过处理器的页表机制实现了从虚拟地址空间到物理地址空间的映射,虚拟化场景的出现在这一基础上又增加了扩展页表EPT机制,额外增加了一级虚拟机物理地址到主机物理地址的映射。之后是内存控制器将处理器的物理地址空间映射到DRAM阵列中,通常使用静态地址映射,通过某种固定的规则将物理地址空间映射到DRAM的bank、row和column中。

在PARD架构中,通过在内存控制器前增加MMU模块,通过映射表将不同应用标签的访存请求进行隔离。但这种方式只实现了一种固定的地址映射机制,即只能进行连续的大块地址分配,无法实现EPT等技术所支持的细粒度内存空间管理。为解决这一问题,可以将静态的MMU模块替换为一个处理器,能够在其中编写软件代码实现地址空间的映射。通过这种方式除了可以完成PARD中MMU的功能外,还可以实现更细粒度的空间管理,也可以实现现有虚拟化平台中常见的基于内容的内存空间压缩机制。除此之外,该处理器还可以通过对请求数据进行额外处理,以在硬件层面实现更复杂的如数据加密、敏感词过滤等高级功能。

访存调度 在PARD的内存控制平面中,只实现了简单的基于优先级的访存调度,可以通过增加处理器的方式实现更为灵活的调度策略。同时也可以像PARDIS[17]工作一样,将处理器加入到内存控制器内部的请求调度模块中,根据不同应用的需求实现不同的DRAM调度策略。

\subsubsection*{缓存控制器}
Cache控制器的功能是对到达的请求进行缓存操作,使用不同的替换策略,对数据访问热度进行预测,以提高访存命中率,提高系统性能。Cache的核心主要包括TagArray、DataArray和替换策略三个部分。以Cache作为数据平面,可以在容量划分与替换策略两个角度增加可编程功能。

容量划分 传统的Cache并没有提供容量划分功能,因此不同应用在共享Cache上运行会造成不同程度的干扰。Intel最新提出的CAT技术[28]在Cache上增加了按路的缓存容量划分机制。但按路划分并非适合所有的应用,可以通过使用处理器替换固定的Set/Way映射方式,根据应用实现更为灵活的缓存容量划分方式。

替换策略 与容量划分的需求类似,不同应用的访存模式不同,固定的缓存替换策略并不能很好的适应所有应用。因此使用处理器与软件替换策略,与PARD的应用区分机制结合,可以实现更为灵活高效的缓存。


\subsection{数据平面处理器体系结构}

由于数据平面处理器位于请求处理的关键路径上,为了保障系统的性能不受影响,需要从以下三个方面进行考虑:
(1)处理器需要执行一系列指令才能完成对请求的处理,因此处理器需要工作在比其所在硬件部件更高频率,以满足硬件部件的性能需求;
(2)处理器的固件代码执行需要确定性,因此不能使用cache结构,而是使用scratchpad memory代替;
(3)由于高频需求,因此处理器功能要尽可能简单,一些必需的复杂的逻辑(如数据压缩与加密等)通过外部加速器的方式进行扩展。
 
基于以上需求,我们选择使用RISC作为数据平面处理器的基础架构,如图3所示。我们首先对传统RISC架构进行精简,只保留其基本功能,以保证其频率需求;同时增加scratchpad memory作为其指令与数据存储,增加请求缓存接口用于接入硬件设备中,增加控制平面接口用于连接控制平面。
从第2 章可知,对于内存控制器与Cache,该处理器的主要工作包括:对请求进行调度、地址变换,对数据进行处理,生成控制信号(如Cache缓存与替换)。要完成以上工作,该处理器需要具备基本的处理器功能外,还需要在数据类型、存储模型和指令上进行扩展。

\subsubsection{数据类型}
数据平面处理器中执行的算法代码大都只是对输入的请求进行处理,因此我们只有“无符号整数”和“请求”两种数据类型,如图4所示。无符号整数的长度与处理器的位宽相同,都被设置为其所属硬件的位宽,以节约请求处理时位宽转换的开销,保障请求处理的效率。
 
“请求”类型的是一个变长数据类型,其中包含了固定的16位应用标签(DSid)以及变长的请求数据。以访存请求为例,其中包含请求地址、长度、线程号、读写类型、锁与缓存状态等其它一些标志位;对于Cache替换请求,其中包含了请求地址、Hit/Miss标记以及其它一些标志位等信息。图4给出了访存请求以及Cache替换请求类型的示例。数据平面处理器本身并不关心请求类型中具体每个位的意义,而只是将其做一个整体进行处理,对每个域的解析或修改由其运行的固件代码完成。

\subsubsection{存储模型}
数据平面处理器中程序员可见的存储结构包含寄存器、请求缓存、scratchpad memory、I/O地址空间四部分。与传统的RISC架构相同,数据平面处理器包含32个通用寄存器(R0-R31),用于进行算数逻辑运算,其中R0是常数零;除此之外,增加了4个用于保存“请求”类型数据的请求寄存器(S0-S3),可以通过请求缓存操作指令(rbget和rbput,参见3.3节),将请求输入队列中的请求读取到该寄存器、或将该寄存器中的请求加入到请求输出队列中;请求寄存器不能直接参与算术逻辑计算,需要先将其部分数据读取到通用寄存器后才能执行计算;请求寄存器之间可以直接进行数据交换。处理器执行的固件代码与数据保存在scratchpad memory中,需要用户自行管理。控制平面被映射为数据平面处理器的外设,提供处理器的固件代码ROM以及处理器的对外接口。

\subsubsection{指令集}
数据平面处理器使用RISC标准的算数逻辑、控制流和访存指令,并在其基础上额外增加了请求缓存操作指令。

请求缓存分为两部分,一个输入缓存一个输出缓存。其中输入缓存既可作为FIFO操作,也可基于DSid进行内容寻址;输出缓存只能作为FIFO操作。用于请求缓存操作的指令如图5所示,指令rbput可以将指定请求寄存器中的请求添加到输出缓存队列末尾;指令rbget有两种使用方式,一种是将输入缓存作为FIFO,取出队列头的请求到请求寄存器,另一种是通过应用标签(DSid)对请求进行筛选,取出第一个满足应用标签的请求到请求寄存器。指令rbcp用于在请求寄存器之间传送数据。指令mfrb用于将请求寄存器中的部分数据传送到通用寄存器;指令mtrb与之相反,用于将通用寄存器的数据传送到请求寄存器指定的位置。

\subsubsection{固件代码示例}
本节将以三段不同功能的固件代码为例,介绍数据平面处理器的编程方法。

\textbf{段式地址映射}\ 本示例用于实现PARD的内存控制器控制平面所提供的地址映射功能,该功能只需要对访存请求的地址进行修改,而请求的数据无需修改,我们将地址与数据分开由两个处理器进行处理,如图6右所示。对于数据处理器,其固件代码只使用rbget/rbput指令对请求进行转发。地址处理器首先需要使用rbget指令获取当前请求到请求寄存器,并使用mfrb指令将其中的地址与DSid读取到通用寄存器;而后通过查表的方式获得该请求对应的映射目的地址的基址,对请求地址进行变换,并使用mtrb指令将变换后的地址写回到请求寄存器,最后通过rbput指令将新的访存请求从处理器中送出,完成地址映射功能。
 
\textbf{数据加密}\ 本示例实现访存数据加密功能,由于数据加密操作通常需要耗费很长的时间,而且我们的数据平面处理器提供的指令集并不足以完成该操作。因此我们在外部实现了硬件AES加密模块,并通过请求接口将其连接到数据平面处理器上,该结构如图7所示。基于该结构,数据处理器只需要将数据发送到AES模块并等待其完成加密,将加密后的数据送出处理器即可。数据解密与加密过程类似,只需将数据发送到连接有解密模块的请求接口即可。
 
\textbf{缓存替换策略}\ 本示例实现缓存替换策略功能,使用可编程处理器替换Cache中原有的LRU模块,使用软件实现基于二叉树的伪LRU替换策略。处理器固件代码工作流程如下:(1)处理器收到Cache前端的请求后,以及Hit/Miss信息,如果缓存命中则无需任何额外操作;(2)对于缓存缺失的请求,首先从地址中解析出tag与set信息,并将解析后的set地址发送到Tag Array,等待其返回该set的信息;(3)根据TagArrary返回的set信息,以及内部的数据结构生成替换目标;(4)将替换目标送出处理器。


\section{小结}

