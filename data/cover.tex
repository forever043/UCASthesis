
%%% Local Variables:
%%% mode: latex
%%% TeX-master: t
%%% End:
\secretcontent{绝密}

\ctitle{面向数据中心应用服务质量保障的资源管理可编程\\体系结构研究}
\makeatletter
\makeatother
\cdegree{工学博士}
\cdepartment[计算所]{中国科学院计算技术研究所}
\cmajor{计算机系统结构}
\cauthor{马久跃} 
\csupervisor{孙凝晖\hspace{1em}研究员}
\csupervisorplace{中国科学院计算技术研究所}

% 日期自动生成,如果你要自己写就改这个cdate
% \cdate{\CJKdigits{\the\year}年\CJKnumber{\the\month}月}

\etitle{Programmable Architecture for Quality-of-Service in Datacenters }
\edegree{Doctor of Philosophy}
\eauthor{Ma Jiuyue}
\edepartment{Institute of Computing Technology, Chinese Academy of Sciences}
\emajor{Computer Architecture}
\esupervisor{Sun Ninghui}

% 这个日期也会自动生成,你要改么?
% \edate{December, 2005}

% 中英文摘要和关键字
\begin{cabstract}
  当前数据中心正面临着平衡资源利用率与应用服务质量的挑战。
  通过负载融合的方式将让来自不同用户的应用共享服务器资源,可以有效的提高资源利用率,
  但这种无管理的软硬件资源共享所产生的资源竞争,会给应用带来不可预测的性能波动。
  而为了保障延迟敏感型在线应用的服务质量,数据中心管理员或开发者会倾向于避免共享,
  使用独占或过量资源预留的方式降低由于共享对在线应用的影响。
  这了造成了数据中心极低的资源利用率。

  该问题产生的根本原因是硬件不能实现应用区分,无法实现应用之间的性能隔离,
  使得共享资源的应用之间产生干扰。
  因此,在应用数量众多、需求多样且不断变化的数据中心场景下,
  计算机体系结构需要重新设计,为应用提供区分化服务、良好的性能隔离,
  并具备灵活的资源管理编程接口,实现资源使用的强控制与按需分配,
  才能解决数据中心资源利用率与服务质量冲突的问题。

  本文围绕数据中心资源利用率与应用服务质量冲突的问题,
  在服务质量保障的体系结构支持方向上开展研究,主要的工作和贡献包括:

  1. 提出``标签化地址空间(Labeled Address Space)''概念。
     正如多进程技术的出现引入了虚拟地址空间抽象,随着虚拟化、云计算与多租户使用模式的出现,
     现有的虚拟地址空间抽象无法满足多租户之间的隔离需求,
     一些硬件隔离技术如EPT、I/O MMU、SR-IOV等试图在现有的体系结构下支持隔离需求,
     其本质则是在虚拟地址空间外增加一层额外的地址空间,但这些技术只是在功能层面上实现了隔离,
     而与性能相关的部件如共享末级缓存、内存控制器等在现有体系结构下并没有实现隔离。
     本文提出的标签化地址空间抽象,使用统一标签区分不同应用,
     并为计算机系统内所有请求标识应用标签,硬件不再需要通过猜测的方式区分应用,
     而是通过标签机制打破目前体系结构中软硬件之间的语言鸿沟,
     使得共享的硬件资源能够区分来自不同应用的请求并进行区分处理。

  2. 提出硬件资源共享管理方法,为硬件资源共享提供配置、监控、反馈功能,
     实现毫秒(ms)级的性能反馈。
     实时的监控与反馈是实现细粒度资源管理的重要前提,软件监控方案无法满足实时性的需求,
     在硬件上直接实现监控与反馈,提出使用通用的``控制平面''实现硬件资源的配置与监控。
     在具体实现上,通过基于表的控制平面实现通用的硬件资源管理接口,
     使用基于处理器的数据平面实现硬件请求的灵活控制,
     为计算机系统实现硬件资源可管理提供支持。

  3. 提出节点内硬件共享资源的协同管理。
     控制平面构成了计算机系统内硬件资源管理的基本单元,由于硬件资源之间具有关联性,
     需要进行全局统筹管理。
     本文通过使用控制面网络将节点内所有的控制平面连接到集中式的资源管理模块,
     对硬件共享资源进行协同管理。
\end{cabstract}

\ckeywords{数据中心, 服务质量, 资源利用率, 虚拟化, 资源管理, 可编程}

\begin{eabstract} 
  Contemporary data centers confront with challenges in
  managing the trade-offs between resource utilization and
  applications’ quality of services (QoS). Co-locate
  multiple workloads into a single server can achieve high
  utilization. But the unmanaged contention for shared
  hardware resources, as well as shared software resources,
  may cause unpredictable performance variability. To
  guarantee QoS of latency-critical online services, data
  center operators or developers tend to avoid sharing by
  either dedicating resources or exaggerating reservations
  for online services in shared environments, which result
  in lower utilization – only 6\% to 12\%.


\end{eabstract}

\ekeywords{Datacenter, QoS}
