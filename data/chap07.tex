
%%% Local Variables:
%%% mode: latex
%%% TeX-master: t
%%% End:

\chapter{PARD原型系统实现与验证}
\label{chap:impl}

%FPGA原型设计目标
上一章已经讨论过如何在X86平台上实现PARD,本章讨论如何在gem5全系统模拟器上将PARD机制
实现其中。验证功能可行性,


\section{基础系统选择}

实现PARD原型系统的第一个工作是选择一个合适的基础系统,
PARD对基础系统有以下需求:

\begin{enumerate}
  \item 能够在FPGA环境下综合;
  \item 独立系统,不依靠任何辅助设施即可运行;
  \item 丰富的I/O设备支持,支持Xilinx VC709开发板所提供的外设,如以太网和PCI-Express;
  \item 能够运行Linux操作系统;
  \item 频率/性能足够运行常见Benchmark应用,如SpecCPU、PARSEC等);
  \item 可以运行典型的数据中心应用,如memcached、httpd等;
  \item 具备完整的软件开发环境;
\end{enumerate}

在开源领域虽然有大量可用的处理器软核,如Oracle OpenSPARC T1\cite{sparct1}、
RISC-V\cite{riscv}、OpenRISC 1200\cite{or1200}、LEON3\cite{leon3}等,
但这些软核并不能满足PARD原型系统的需求。
其中OpenSPARC T1和RISC-V目前的FPGA实现并不是一个独立系统,
需要额外的处理器(如MicroBlaze或ARM)作代理以实现访存与I/O操作;
OpenRISC 1200的软硬件环境支持并不完整,
LEON3是目前开源的处理器软核中最为合适的选择,但其对linux内核的支持并不好,
目前只能运行早期的内核版本,同时软件环境也比较老旧,运行数据中心应用存在一定的困难。
一些FPGA厂商也提供了可配置的处理器核,
如Xilinx的MicroBlaze\cite{microblaze}和ARM\cite{zynq},以及Altera的NIOS II\cite{niosii}。

本文最终选择了Xilinx的MicroBlaze作为基础系统,
其性能





\subsection{CPU/System}
\subsection{DSid集成}
\subsection{AMP系统实现}

%包含踩过的坑,如基础系统选择,最初的OpenSPARC方案+寄存器修改 => MicroBlaze外包一层


\section{PARD原型系统架构}


\section{PRM与控制平面网络设计}

为何选择I2C,有何替代

如何改造I2C成为双向网络

连接多个控制平台

触发表中断传播机制


\section{系统设计与关键技术}

\subsection{标签集成}

\subsection{通用控制平面设计}

\subsubsection*{表设计}
\subsubsection*{状态表更新逻辑}
\subsubsection*{触发表逻辑}

\subsection{通用数据平面设计}

\subsection{系统集成}

\subsubsection*{系统集成:处理器核}

\subsubsection*{系统集成:共享末级缓存}

\subsubsection*{系统集成:内存控制器}

\subsubsection*{系统集成:I/O子系统}


\subsection{小结}
包含踩过的坑,如基础系统选择,最初的OpenSPARC方案+寄存器修改 => MicroBlaze外包一层
FPGA实现中遇到的问题


\section{软件栈设计}



\section{性能评估}

\subsection{区分化服务}

\subsection{性能隔离}

\subsection{策略生效时间}

并讨论如何实现ms级控制

\subsection{实际应用测试}

\subsection{开销}


\section{小结}


\if 0	% 模拟器移动到第4、5章相应的章节

\section{模拟器实现}
\subsection{基本配置}

\subsection{模拟器设计目标}

与网络/SDN对比的功能表格,列出实现目标

列出模拟器局限

\subsection{系统设计与关键技术}

\subsubsection{实现全硬件虚拟化}

标签机制 + core/LLC/mc/io
ptable + MultiOS

\subsubsection{实现资源预留}

stable统计 + LLC partition

\subsubsection{实现QoS}

ttable + scheduling + feedback

\subsubsection{管理}

PRM + Program/Trigger

\subsubsection{小结}

总结已实现的功能,并对未来可能扩展的功能进行展望

\subsection{小结}

\fi

