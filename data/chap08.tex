
%%% Local Variables:
%%% mode: latex
%%% TeX-master: t
%%% End:

\chapter{结束语}
\label{cha:concl}

\section{本文工作总结}
This paper presented a study of building DiffServ in computer
servers via architectural support. We proposed programmable
architecture for resourcing-on-demand(PARD) and developed a
Linux-based firmware to facilitate programming PARD. We implemented
PARD on both a full-system simulator and an FPGA
development board. Our experiments demonstrated that PARD is
able to address the trade-offs between high utilization and high
QoS in datacenter environments.


\section{下一步研究方向}

We believe that architectural support for DiffServ in computers
is a promising trend as the number of cores keeps increasing. PARD
makes a case for this direction and provides new interfaces for users
to interact with the hardware. However, there are still a lot of open
issues such as

how to translate applications’ QoS requirements into efficient
“trigger)action” rules?

how to leverage compilers to automatically generate trigger
rules?

how to make OS directly run on PARD server to support
process-level DiffServ?

how to support fine-grain Diffserv for not only process-level but
also thread-level or even C++/Java object-level?

how to support nested DiffServ, i.e., guarantee QoS of a process
within a LDom?
 
how to deal with simultaneous multithreading (SMT)?
 
how to deal with multiple processors issues such as cache coherency?

how to extend DiffServ to accelerators, e.g., enabling an encryption
engine to encrypt/decrypt LDoms with specific DS-id?

how to integrate PARD and SDN so that DS-id can be propagated
in a data center wide?

how to design and deploy security policy on PARD servers?

how to develop firmware applications to take advantage of
PARD?

To facilitate further optimization and exploration, we
release the PARD-gem5 simulator that is available at
https://github.com/fsg-ict/PARD-gem5.

