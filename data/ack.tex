%%% Local Variables:
%%% mode: latex
%%% TeX-master: "../main"
%%% End:

\begin{ack}

转瞬相识16年,初时懵懂的玩伴,已成为如今奋斗的战友。
还记得与你初识,CAI和LOGO是我们话题,小乌龟是我们共同的玩伴;
似懂非懂的啃完《数据结构:C语言描述》,才发现你我之间的距离,很远;
16年的不离不弃,你更加强大,我开始懂你。
能够为你的家族增加一个新的孩子PARD,已无遗憾。

%博士是一个很神奇的
%自己懂的越多,对前辈们的敬仰越深,

首先要感谢我导师孙凝晖老师,成为您的学生是一个偶然,也是一个幸运。

还要感谢那些在重要的人生选择时给予我帮助的人:
感谢许强老师,是您将我从书本和实验带入到真正的项目,教会我需求分析、项目管理,这是我最宝贵的财富;
感谢董天正教授,是您将我引领到计算机系统结构领域,并让我对它产生浓厚的兴趣;
感谢熊劲老师,您严谨的的治学态度,是我一直学习与坚持的榜样;
感谢李卓坚老师,是您手把手的教我认识机房与服务器,引领我进入系统管理的领域;
感谢马捷老师,虽然没能成为您的学生,但您教会了我规范的重要。
特别要感谢包云岗老师,您是我的指路人,为我迷茫的博士路指明了方向,
如果没有您的指导与帮助,
更不会有机会去完成自己造一台计算机的梦想。

感谢与我共同奋斗的师弟师妹们:
感谢余子濠(濠神),把PARD交给你我放心;
感谢黄博文(布布)、靳鑫(鑫娃子),你们是我硬件入门的老师;
感谢展旭升(我展)、徐天妮(我妮)、姚治成(姚姚)、屈雨鹏(XP),
和大家一起做科研是非常Happy的事情。
感谢张子刚,一起奋斗在博士的最后阶段,没有战友,一个人的战斗会非常的艰辛。

感谢我的妻子王一帆,感谢你一直陪在我的身边,包容我的所有缺点,
在我低落的时候给我鼓励,在我成功的时候为我高兴,
这些和你准备的爱心午餐,都是我能够坚持的动力。

仅以此文,献给我的英雄父亲。
\newline\newline

\rightline{2016年05月21日}
\rightline{于计算所\qquad}

\end{ack}
