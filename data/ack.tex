%%% Local Variables:
%%% mode: latex
%%% TeX-master: "../main"
%%% End:

\begin{ack}

转瞬在计算所读博的七年即将过去,这时才猛然发现,
和计算机相识已经16年了。
还记得初识时的CAI和LOGO,
还有那本似懂非懂的《数据结构:C语言描述》,
是我走上计算机道路的开始。
在这里要感谢那些在路上给予我关怀、指导和帮助的人们。

首先要感谢我的导师孙凝晖老师,成为您的学生我的幸运。
您渊博的知识、严谨的治学态度和敏锐的思维,还您对中国计算机事业的使命感,
是我今后学习的榜样。

还要感谢那些在重要的人生选择时给予我帮助的人:
感谢许强老师,是您将我从书本和实验带入到真正的项目,教会我需求分析、项目管理;
感谢董天正教授,是您带我进入到计算机系统结构领域;
感谢熊劲老师,您严谨的的治学态度,是我一直学习与坚持的榜样;
感谢李卓坚老师,是您手把手的教我认识机房与服务器,引领我进入系统管理的领域;
感谢马捷老师,虽然没能成为您的学生,但您教会了我规范的重要。
特别要感谢包云岗老师,您是我的指路人,为我迷茫的博士路指明了方向,
也是在您的帮助下,我完成自己进入计算所时的梦想,造出了属于自己的计算机。

感谢计算所智能中心、高性能中心和先进计算机系统研究中心对我的培养。
感谢已经毕业的师兄们,
特别要感谢邢晶师兄、马灿师兄和李强师兄,是你们教会了我坚持与信念。
感谢与我共同奋斗的师弟师妹们:
感谢余子濠(濠神),我大PARD的重任就交给你了;
感谢黄博文、靳鑫,你们是我硬件入门的老师;
感谢展旭升、李宇鹏、徐天妮、姚治成、屈雨鹏、李文捷,和大家一起做科研是非常愉快的事情。
感谢张子刚,一起奋斗在博士的最后阶段,没有战友,一个人的战斗会非常的艰辛。

感谢徐志伟老师、冯晓兵老师、陈云霁老师、谢源老师、
孙广宇老师对我的学位论文提出了宝贵的修改意见。
感谢Donald E. Knuth以及Leslie Lamport,
他们开发的\TeX 和\LaTeX 使我可以专注于论文的写作。
还要感谢\ucasthesis 及其维护者xiaoyao9933,
它让我的论文写作轻松自在了许多,让我的论文格式规整漂亮了许多。

最后要感谢我的家人,感谢我的妈妈,
是您的关心与支持让我能够毫无顾虑的向着自己的理想奋斗,
您对我的爱一直是我前进的动力。

感谢我亲爱的妻子王一帆,一直默默的陪在我身边支持我、鼓励我,
是你的付出与理解,还有那些精心准备的爱心午餐,
让我能够专心科研、顺利毕业。

仅以此文,献给我的父亲。
\newline\newline

\rightline{2016年05月21日\qquad}
\rightline{于计算所\qquad\qquad}

\end{ack}
