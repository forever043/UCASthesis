
%%% Local Variables:
%%% mode: latex
%%% TeX-master: t
%%% End:

\chapter{资源管理可编程体系结构}
\label{cha:pard}

上一章实验结果表明在Cache层次上区分应用并做划分,可以取得一些效果;但系统中还存在其它
一些共享资源,如MC、I/O等,也存在竞争问题,如何在这些位置做管理?
同时可控制的资源变多之后,如何管理这些资源,让他们协同工作,也是个问题。
本章将提出一种全新的体系结构PARD,以解决上面提出的问题。

本章首先总体介绍PARD体系结构的特点,然后分析如何将一个现有的系统改造成PARD系统,
并从用户角度说明PARD能够使用的功能,最后分析PARD在现有X86/ARM平台上实现的可行性。


\section{Overview}

什么是资源管理可编程体系结构,关键点:
 * Tag机制,实现Labeled Address Space / Performance Space
 * 可编程机制,

%\subsubsection{Key Concept of PARD}
%Key1: Labeled Address Space:机器可划分,细粒度使用
%Key2: 将SDN引入体系结构:中心化管理,可编程

\section{How to construct a PARD server}

如上节所述,PARD并不是一个完全全新的体系结构,而是对现有体系结构的扩展。
为了说明如何将PARD扩展到一个现有的体系结构中,本节首先构建了一台虚拟计算机,
我们将其命名为XXXX,如图\ref{fig:XXX-computer}所示。
XXX包含两路4核的处理器,每个处理器核拥有独立的一级缓存L1-I和L1-D,
每个Socket的四个处理器核共享一个16路2MB的二级缓存;
两个Socket拥有自己的内存控制器,同时使用MESI目录协议实现NUMA内存访问;
XXX的I/O子系统包含SATA控制器和两个以太网卡,通过IOH芯片与两个处理器相连。
XXX的结构可以很好的匹配到目前流行体系结构中(如X86或ARM)。

\subsection{Computer as a Network => PARD}

\subsection{像SDN一样集中式的管理计算机}


\section{PARD的用户视图}

关键功能+使用方法+流程示意


\section{体系结构的可行性}

在X86或ARM上实现该架构


\section{小结}


