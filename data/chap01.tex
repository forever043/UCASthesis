
%%% Local Variables:
%%% mode: latex
%%% TeX-master: t
%%% End:

\chapter{引言}
\label{chap:intro}

互联网与云计算的发展,越来越多的应用从本地迁移到云端,并涌现出大量新兴的互联网应用,
承载这些应用的数据中心成为了如同电力系统一样的社会基础设施。
与此同时,现代数据中心正在面临着权衡资源利用率与应用服务质量的挑战:
从应用开发者角度,服务质量是第一位的,因为它直接关系到用户体验与其收益,
由于互联网应用负载的波动性,开发人员通常会为自己的应用过量分配资源以满足峰值时的负载需求,
这造成了非常低的服务器利用率,通常只有6\%-12\%;
而对于数据中心运维人员,资源利用率直接反映其运维成本,虽然将不同应用混合部署到同一台服务器,
充分利用空闲时段的服务器资源可以有效提高数据中心利用率,
但多应用混合部署引入的软硬件资源共享会造成应用间无管理的资源竞争,
使得应用性能出现不可预测的波动,进而影响应用的服务质量。
由上可知,如何权衡数据中心资源利用率与应用服务质量是当前数据中心亟待解决的重要问题。

可以从三个角度解决这一问题:
其一是通过上层软件机制实现干扰容忍,在应用层保障服务质量,
如Google提出的Hedged Requests和Tied Requests方案\cite{tailatscale2013},
通过向多个副本发送请求,并选择最快返回的结果以达到干扰容忍的目的。
其二是在应用调度层次,通过profile的方式预测应用混合后的干扰情况,
将相互之间干扰较小的应用部署到同一台服务器。
其三是提供一个良好的隔离环境,降低由资源竞争所产生的应用间干扰,
可以在各个层次实现隔离,如操作系统级\cite{cgroup}、Hypervisor级\cite{}、硬件级\cite{}。

其中前两个方案在实施时需要对目标应用具有非常深入的理解:
方案一需要对应用架构与实现细节进行修改,以达到应用层干扰容忍的目的;
方案二虽然不需要对应用进行修改,但需要对应用的资源占用以及不同应用之间的干扰状况进行深入分析,
才能得到最优的调度方案。
当应用数量不多且有条件进行以上所述的分析和修改,通过精细的应用架构设计与调度机制,
可以有效的解决前文所提到的资源利用率与服务质量相冲突的问题。
但在现实数据中心特别是云计算数据中心内,以上假设并不成立。

首先,数据中心内通常会运行大量的应用,如Google的数据\cite{}表明其数据中心在两个月内共运行了超过2000000个应用,
无论是改造这些应用或是对应用之间的干扰行为进行分析都是不可行的。
即使只对部分关键的应用进行改造使其适应干扰环境,云计算环境下的“吵闹的邻居(Noisy Neighbors)”也会使这些努力的结效果大打折扣。

其次,调度方案无法解决短时运行的干扰应用对其它正常应用带来的影响,特别是随着DevOps的兴起,
由于开发调试与线上部署是不断迭代进行的,调试过程中所引入的短时干扰应用数量大大增加,
Google的数据\cite{}发现大量的小于6分钟的应用都是来自于这些调试应用。
当调度器发现干扰并准备采取调度措施时干扰应用可能已经结束,同时新的干扰应用又开始运行,并带来新的干扰。

%% 需要增加对软硬件隔离的综述
%% 为什么隔离才是最佳选择?
综上可知,在当前云计算数据中心中隔离是解决资源利用率与服务质量的最佳选择。



现有数据中心技术面临资源利用率与应用服务质量的矛盾,其根本原因是大量数据中心共享资源
属于“无管理共享”状态。要实现高效通用数据中心目标,核心是从硬件上改变资源的“无
管理共享”现状以实现在体系结构上支持应用服务质量保障,在此基础上实现数据中心资源根据应
用动态管理以提高资源利用率。

回顾历史,当前数据中心面临的问题与1990年代的Internet具有相似之处。
当时流媒体、电子商务、电子邮件和FTP等大量网络应用的兴起,它们具有不同的QoS需求。
为了在IP网络中为这些应用提供端到端的服务质量保障,
互联网工程任务组(Internet Engineering Task Force, IETF)于1998年提出了区分化服务
(Differentiated Services)的概念。而如今,区分化服务已经成为应用最广泛的服务质量保障
机制之一。该技术的核心是在IP包头中定义长度为8-bit的区分化服务域,用以表示应用的
服务质量分类标识,因此路由器、交换机等网络设备便可以使用该信息对不同类别的数据包
进行区分处理,以达到区分化服务的目的。
软件定义网络(SDN)的出现,进一步促进了网络领域服务质量保障的发展,
其主要原理可以概括为:(1) 控制平面与数据平面分离;(2)集中控制的统一编程接口。

同时,计算机内部也可以被看做一个网络,如图\ref{fig:computer-as-a-network}所示,
CPU核、共享缓存、内存控制器、I/O设备等可以被看做是网络节点;除了处理请求以外,
这些“网络节点”与网络中的路由器/交换机具有相似的请求转发功能;
而它们之间也通过包进行通信,如:片内通信使用NoC包,片间通信的QPI/HT包,
以及I/O部分使用的PCI-E包。
将网络领域的区分化服务和软件定义网络的思想应用到计算机内部的网络,
用以解决数据中心当前面临的资源利用率与应用服务质量矛盾,是本文的主要研究思路与动机。
