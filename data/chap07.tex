
%%% Local Variables:
%%% mode: latex
%%% TeX-master: t
%%% End:

\chapter{PARD原型系统实现与验证}
\label{chap:impl}

上一章已经讨论过如何在X86平台上实现PARD,本章讨论如何在gem5全系统模拟器上将PARD机制
实现其中。验证功能可行性,

\section{模拟器实现}
\subsection{基本配置}

\subsection{模拟器设计目标}

与网络/SDN对比的功能表格,列出实现目标

列出模拟器局限

\subsection{系统设计与关键技术}

\subsubsection{实现全硬件虚拟化}

标签机制 + core/LLC/mc/io
ptable + MultiOS

\subsubsection{实现资源预留}

stable统计 + LLC partition

\subsubsection{实现QoS}

ttable + scheduling + feedback

\subsubsection{管理}

PRM + Program/Trigger

\subsubsection{小结}

总结已实现的功能,并对未来可能扩展的功能进行展望

\subsection{小结}


\section{PARD FPGA原型系统}

\section{FPGA原型设计目标}


\section{系统设计与关键技术}

\subsection{基础系统选择}

\subsubsection*{CPU/System}
\subsubsection*{DSid集成}
\subsubsection*{AMP系统实现}


\subsection{PRM系统设计}



\subsection{CPN总线选择}

为何选择I2C,有何替代

如何改造I2C成为双向网络

连接多个控制平台

触发表中断传播机制


\subsection{控制平台设计变化}

\subsubsection*{表设计}
\subsubsection*{状态表更新逻辑}
\subsubsection*{触发表逻辑}


\subsection{性能控制相关}
\subsubsection*{Core修改与控制面设计}
\subsubsection*{Cache修改与控制面设计}
\subsubsection*{MIG修改与控制面设计}

\subsection{小结}
包含踩过的坑,如基础系统选择,最初的OpenSPARC方案+寄存器修改 => MicroBlaze外包一层
FPGA实现中遇到的问题

\subsection{FPGA原型的最终形态}


\section{性能评估}

\subsection{区分化服务}

\subsection{性能隔离}

\subsection{策略生效时间}

并讨论如何实现ms级控制

\subsection{实际应用测试}

\subsection{开销}


\section{小结}

