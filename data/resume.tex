\begin{resume}

\noindent
姓名:马久跃  性别:男  出生日期:1988.10.19  籍贯:辽宁\\

\noindent
2009.9 -- 现在       中国科学院计算技术所 计算机体系结构专业硕博研究生

\noindent
2005.9 -- 2009.7      东北师范大学软件学院 本科生\\

  \resumeitem{攻读博士学位期间发表的论文}
  \begin{enumerate}[leftmargin=1.5\parindent, nolistsep, label={[\arabic*]}]
    \item \textbf{Jiuyue Ma}, Xiufeng Sui, Ninghui Sun, Yupeng Li, Zihao Yu, Bowen Huang, Tianni Xu, Zhicheng Yao, Yu Chen, Haibin Wang, Lixin Zhang, Yungang Bao.
          Supporting Differentiated Services in Computers via Programmable Architecture for Resourcing-on- Demand (PARD)[C]. ASPLOS'15.
    \item \textbf{马久跃}, 余子濠, 包云岗, 孙凝晖. 体系结构内可编程数据平面方法[J]. 计算机研究与发展.(已录用)
    \item \textbf{Jiuyue Ma}, Xiufeng Sui, Yupeng Li, Zihao Yu, Bowen Huang, Yungang Bao, Supporting Differentiated Services in Datacenter Servers[C]. OSDI'14 Poster.
    \item Rui Ren, \textbf{Jiuyue Ma}, Xiufeng Sui, Yungang Bao. D2P: a distributed deadline propagation approach to tolerate long-tail latency in datacenters[C]. APSys'14.
  \end{enumerate}

  \resumeitem{专利}
  \begin{enumerate}[leftmargin=1.5\parindent, nolistsep, label={[\arabic*]}]
    \item \textbf{马久跃},刘立坤,严得辰,李旭;基于设备能力的多终端数据同步方法和系统;申请号:201210208518.5;授权
    \item \textbf{马久跃},姜继,陈克平,熊劲;一种虚拟化环境下用户数据的读写方法、系统及物理机;申请号:201210572237.8;公开
    \item \textbf{马久跃},包云岗,隋秀峰,任睿;一种请求处理方法、装置及系统;申请号:201310228246X;公开
    \item \textbf{马久跃},包云岗,任睿,隋秀峰;控制方法和控制设备;申请号:CN201410181775.3;公开
    \item 包云岗,\textbf{马久跃},隋秀峰,任睿,张立新;计算机,控制设备和数据处理方法;CN201410682375.0;公开
  \end{enumerate}

  \resumeitem{攻读博士学位期间参加的科研项目}
  \begin{enumerate}[leftmargin=1.5\parindent, nolistsep, label={[\arabic*]}]
    \item 华为-计算所合作项目``高通量服务器前瞻课题'',2013年$\sim$2015年
    \item 华为-计算所合作项目``第三代数据中心DC3.0关键技术'',2015年$\sim$2016年
    \item 国际合作重点项目``高效通用数据中心体系结构研究'',2015年$\sim$2019年
  \end{enumerate}

  \resumeitem{攻读博士学位期间的获奖情况}
  \begin{enumerate}[leftmargin=1.5\parindent, nolistsep, label={[\arabic*]}]
    \item 2011年被评为中国科学院``三好学生''
    \item 2012年获曙光博士奖学金
    \item 2015年获博士国家奖学金
  \end{enumerate}
\end{resume}








